
	Antibiotic resistance constitutes a critical global health threat, with World Health Organization projections indicating potential annual mortality in the millions by 2050 if current trends persist. This project develops a mathematical framework for modeling bacterial population dynamics within the human body, specifically examining the tripartite interaction between pathogenic bacteria, host immune response, and antibiotic interventions. 
	
	\subsection{Problem Context}
	Traditional models of bacterial dynamics exhibit a significant limitation: they typically isolate either antibiotic effects \textit{or} immune response mechanisms. This fragmentation neglects the synergistic reality where:
	\begin{enumerate}
		\item \textbf{Antibiotic-resistant bacteria (ARB)} employ survival mechanisms including target-site modification and plasmid-mediated resistance gene transfer
		\item \textbf{Non-resistant bacteria} compete for resources while responding differently to selective pressures
		\item \textbf{Immune effectors} dynamically interact with both bacterial subpopulations
		\item \textbf{Antibiotics} simultaneously impose selective pressure while modulating immune activity
	\end{enumerate}
	The clinical urgency is underscored by resistance drivers such as antibiotic overuse (e.g., inappropriate viral infection treatment), premature therapy discontinuation, and incorrect dosing—all contributing to evolutionary selection favoring resistant strains.
	
	\subsection{Modeling Imperatives}
	Treatment complexity escalates dramatically with resistance emergence due to:
	\begin{itemize}
		\item Diminished therapeutic options with higher toxicity profiles
		\item Immune function degradation during severe infections
		\item \textbf{Selective pressure} phenomena where indiscriminate antibiotic use eliminates susceptible bacteria while creating ecological niches for ARB expansion
	\end{itemize}
	Epidemiological evidence suggests that modest reductions (e.g., 10\%) in unnecessary antibiotic usage could significantly curb resistance propagation. However, predicting intervention outcomes requires sophisticated modeling accounting for:
	\begin{equation*}
		\underbrace{\text{Bacterial growth kinetics}}_\text{Competition} + \underbrace{\text{Resistance transmission}}_\text{Horizontal transfer} + \underbrace{\text{Immune recruitment}}_\text{Nonlinear dynamics}
	\end{equation*}
	
	\subsection{Proposed Framework}
	Our novel dynamical system addresses existing gaps by concurrently integrating:
	\begin{itemize}
		\item Dual bacterial subpopulations (resistant vs. susceptible)
		\item Adaptive immune response dynamics
		\item Pharmacokinetic/pharmacodynamic antibiotic profiles
	\end{itemize}
	This tripartite model enables quantitative investigation of:
	\begin{itemize}
		\item Equilibrium conditions governing pathogen clearance
		\item Resistance dominance thresholds
		\item Chronic infection persistence criteria
	\end{itemize}
	Through numerical analysis and computational implementation, we examine critical scenarios including:
	\begin{itemize}
		\item Optimal dosing windows minimizing resistance selection
		\item Immune augmentation strategies complementing antibiotics
		\item Failure modes under subtherapeutic drug concentrations
	\end{itemize}
	
	\subsection{Structural Overview}
	This paper proceeds as follows: Section 2 formalizes key biological mechanisms; Section 3 develops the governing equations; Section 4 analyzes equilibrium states; Section 5 presents numerical simulations; and Section 6 details the open-source implementation. Our integrated approach provides a critical tool for simulating combination therapies and optimizing antimicrobial stewardship policies.
